% Created 2021-09-25 Sat 17:16
% Intended LaTeX compiler: pdflatex
\documentclass[11pt]{article}
\usepackage[utf8]{inputenc}
\usepackage[T1]{fontenc}
\usepackage{graphicx}
\usepackage{grffile}
\usepackage{longtable}
\usepackage{wrapfig}
\usepackage{rotating}
\usepackage[normalem]{ulem}
\usepackage{amsmath}
\usepackage{textcomp}
\usepackage{amssymb}
\usepackage{capt-of}
\usepackage{hyperref}
\author{Daniel Ballesteros-Chávez}
\date{\today}
\title{Lecture 1}
\hypersetup{
 pdfauthor={Daniel Ballesteros-Chávez},
 pdftitle={Lecture 1},
 pdfkeywords={},
 pdfsubject={},
 pdfcreator={Emacs 26.1 (Org mode 9.3.6)}, 
 pdflang={English}}
\begin{document}

\maketitle
\tableofcontents


\section{Introduction to R}
\label{sec:orgfb2cae1}

\begin{itemize}
\item What is R?
\end{itemize}

R is an open source statistical software.

\begin{itemize}
\item How to get it?
\end{itemize}

You can get R directly on their website \url{https://www.r-project.org/}

\begin{itemize}
\item What is RStudio?
\end{itemize}

RStudio is an integrated development environment (IDE) for R. 

It is a good tool to get started if you are not used to scripting programming. It has very helpful visual features.

I use \href{https://www.gnu.org/software/emacs/}{emacs} and it is difficult for me to get used to Rstudio. My advice: use whatever you find more comfortable. 



\begin{itemize}
\item Basic commands
\begin{itemize}
\item getwd()
\item setwd()
\item install.packages()
\item library()
\item df <- read.dbf()
\item df\$NAME
\item dim(df)
\item class(df)
\item ls()
\item rm(list=ls())
\item sum(df\$NAME)
\item as.numeric()
\item as.character()
\item df\$NEWVAR <- df\$NAME
\item table()
\item tapply, sapply, lapply, etc.
\end{itemize}
\end{itemize}


\section{The Working Directory}
\label{sec:org6b037e8}

Get my Working Directory (a.k.a. where am I?)
\begin{verbatim}
getwd()
\end{verbatim}


Set my Working Directory (a.k.a. change directory)
\begin{verbatim}
setwd("/home/ennaniux/Documents/R_Modelling")
\end{verbatim}

How does this work in Windows?
\begin{verbatim}
setwd("C:/home/ennaniux/Documents/R_Modelling")
\end{verbatim}

or maybe 
\begin{verbatim}
setwd("C:\\home\\ennaniux\\Documents\\R_Modelling")
\end{verbatim}


\section{Installing packages}
\label{sec:org4e64b8d}

The Packages are sets of tools that can be downloaded from different
servers around the world. Different packages have different R functions for specific purposes.

For example, the foreign package allows R to read different data set files like .sav, .dbf, and other file extensions.

In order to install the package foreign we type in the console
\begin{verbatim}
install.packages("foreign")
\end{verbatim}

this will provide a list of possible servers to choose from, and you
can select one close to your location.

\section{Simple manipulations; numbers and vectors}
\label{sec:orgcbc5e40}

The simplest data structure R operates on is the \textbf{numeric vector}, which
is a single entity consisting of an ordered collection of numbers.

\begin{verbatim}
x <- c(1,3,5,9)
x
\end{verbatim}

The syntax is equivalent to 
\begin{verbatim}
c(1,3,5,9) -> y
y
\end{verbatim}

and 
\begin{verbatim}
assign("z",c(1,3,5,9) )
z
\end{verbatim}

\section{Writing a data frame}
\label{sec:orgda2cfae}

Write a data frame by specifying the columns:

\begin{verbatim}
df <- data.frame(
"NAME" =  c("Aleksandra", "Hugo", "Piotr", "Ewa"),
"AGE"  =  c(29,35, 39, 33),
"HEIGHT"= c(1.68, 1.83, 2.03, 1.66) )
df
\end{verbatim}



What is the dimension of the data frame?
\begin{verbatim}
dim(df)
\end{verbatim}

What are the variable names of the data frame?
\begin{verbatim}
names(df)
\end{verbatim}


\section{Reading a data frame}
\label{sec:org2c6a6a3}

From a .csv file

\begin{verbatim}
df <- read.csv('./path_to/file.csv')
\end{verbatim}



From a .dbf file
\begin{verbatim}
library(foreign)
df <- read.csv('./path/to/file.dbf')
\end{verbatim}

From a .sav file
\begin{verbatim}
library(foreign)
     df <- read.spss(file='./path/to/file.sav', to.data.frame=TRUE) 
     str(df)   # show the structure of the data frame
\end{verbatim}


From the clipboard
\begin{verbatim}
df2 <- read.table(file = "clipboard", sep = "\t", header=TRUE)
\end{verbatim}

\section{Creating a new variable}
\label{sec:org82d570f}

\begin{itemize}
\item Graphics
\item Reading data
\item Markdown
\end{itemize}
\end{document}