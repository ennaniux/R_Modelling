% Created 2021-10-04 Mon 19:20
% Intended LaTeX compiler: pdflatex
\documentclass[11pt]{article}
\usepackage[utf8]{inputenc}
\usepackage[T1]{fontenc}
\usepackage{graphicx}
\usepackage{grffile}
\usepackage{longtable}
\usepackage{wrapfig}
\usepackage{rotating}
\usepackage[normalem]{ulem}
\usepackage{amsmath}
\usepackage{textcomp}
\usepackage{amssymb}
\usepackage{capt-of}
\usepackage{hyperref}
\author{Daniel Ballesteros-Chávez}
\date{\today}
\title{Lecture 1}
\hypersetup{
 pdfauthor={Daniel Ballesteros-Chávez},
 pdftitle={Lecture 1},
 pdfkeywords={},
 pdfsubject={},
 pdfcreator={Emacs 26.1 (Org mode 9.3.6)}, 
 pdflang={English}}
\begin{document}

\maketitle
\tableofcontents




\section{Introduction to R}
\label{sec:org04ada71}

\begin{itemize}
\item What is R?
\end{itemize}

R is an open source statistical software.

\begin{itemize}
\item How to get it?
\end{itemize}

You can get R directly on their website \url{https://www.r-project.org/}

\begin{itemize}
\item What is RStudio?
\end{itemize}

RStudio is an integrated development environment (IDE) for R.

It is a good tool to get started if you are not used to scripting
programming. It has very helpful visual features.

I use \href{https://www.gnu.org/software/emacs/}{emacs} and it is difficult for me to get used to Rstudio. My
advice: use whatever you find more comfortable.



\begin{itemize}
\item Basic commands
\begin{itemize}
\item getwd()
\item setwd()
\item install.packages()
\item library()
\item df <- read.dbf()
\item df\$NAME
\item dim(df)
\item class(df)
\item ls()
\item rm(list=ls())
\item sum(df\$NAME)
\item as.numeric()
\item as.character()
\item df\$NEWVAR <- df\$NAME
\item table()
\item tapply, sapply, lapply, etc.
\end{itemize}
\end{itemize}


\section{The Working Directory}
\label{sec:orgce8b3b4}

Get my Working Directory (a.k.a. where am I?)
\begin{verbatim}
getwd()
\end{verbatim}


Set my Working Directory (a.k.a. change directory)
\begin{verbatim}
setwd("/home/ennaniux/Documents/R_Modelling")
\end{verbatim}

How does this work in Windows?
\begin{verbatim}
setwd("C:/home/ennaniux/Documents/R_Modelling")
\end{verbatim}

or maybe 
\begin{verbatim}
setwd("C:\\home\\ennaniux\\Documents\\R_Modelling")
\end{verbatim}


\section{Installing packages}
\label{sec:org00bf062}

The Packages are sets of tools that can be downloaded from different
servers around the world. Different packages have different R
functions for specific purposes.


The Comprehensive R Archive Network (CRAN)
\url{https://cran.r-project.org/}. CRAN is a network of ftp and web servers
around the world that store identical, up-to-date, versions of code
and documentation for R.

For example, the foreign package allows R to read different data set
files like .sav, .dbf, and other file extensions.

In order to install the package foreign we type in the console
\begin{verbatim}
install.packages("foreign")
\end{verbatim}

this will provide a list of possible servers to choose from, and you
can select one close to your location.



\section{Simple manipulations; numbers and vectors}
\label{sec:org54b4afa}

The simplest data structure R operates on is the \textbf{numeric vector}, which
is a single entity consisting of an ordered collection of numbers.

\begin{verbatim}
x <- c(1,3,5,9)
x
\end{verbatim}

\begin{center}
\begin{tabular}{r}
1\\
3\\
5\\
9\\
\end{tabular}
\end{center}

The syntax is equivalent to 
\begin{verbatim}
c(11,31,15,19) -> y
y
\end{verbatim}

\begin{center}
\begin{tabular}{r}
11\\
31\\
15\\
19\\
\end{tabular}
\end{center}

and 
\begin{verbatim}
assign("z",c(-1,33,2.5,9) )
z
\end{verbatim}

\begin{center}
\begin{tabular}{r}
-1\\
33\\
2.5\\
9\\
\end{tabular}
\end{center}


Linear operations:
\begin{verbatim}
z * y + x
\end{verbatim}

\begin{center}
\begin{tabular}{r}
-10\\
1026\\
42.5\\
180\\
\end{tabular}
\end{center}

Definition of a sequence

\begin{verbatim}
3:10
\end{verbatim}

\begin{center}
\begin{tabular}{r}
3\\
4\\
5\\
6\\
7\\
8\\
9\\
10\\
\end{tabular}
\end{center}

If we want to know the number of entries in our vector, then we use the function \texttt{length}
\begin{verbatim}
length(c(3:10))
\end{verbatim}

\begin{verbatim}
8
\end{verbatim}



R tries to apply operations on vectors
\begin{verbatim}
x <- c(3:10)
x^2
\end{verbatim}

\begin{center}
\begin{tabular}{r}
9\\
16\\
25\\
36\\
49\\
64\\
81\\
100\\
\end{tabular}
\end{center}

The sum of the first 100 natural numbers, if we want to perform the sum

one can do for instance
\begin{verbatim}
x <- c(1:100)
sum(x)
\end{verbatim}

\begin{verbatim}
5050
\end{verbatim}



Missing values are denoted by \texttt{NA}. Whenever there is a missing value, the default behaviour is to be reminded:
\begin{verbatim}
x <- c(4, 4, NA, 2, 3, NA, 5)
sum(x)
\end{verbatim}

\begin{verbatim}
nil
\end{verbatim}


Then, if you want to omit the missing values in the operation you have to specify it
\begin{verbatim}
x <- c(4, 4, NA, 2, 3, NA, 5)
sum(x, na.rm=TRUE)
\end{verbatim}

\begin{verbatim}
18
\end{verbatim}


A vector can also consists of strings or character type entries:
\begin{verbatim}
x <- c("Uno", "Dos", NA, "Cuatro", "Dos", "Cuatro")
unique(x)
\end{verbatim}

\begin{center}
\begin{tabular}{l}
Uno\\
Dos\\
nil\\
Cuatro\\
\end{tabular}
\end{center}


How to we know if we have a missing value?
\begin{verbatim}
x <- c(4, 4, NA, 2, 3, NA, 5)
is.na(x)
\end{verbatim}

\begin{center}
\begin{tabular}{l}
FALSE\\
FALSE\\
TRUE\\
FALSE\\
FALSE\\
TRUE\\
FALSE\\
\end{tabular}
\end{center}


\section{Pre-loaded data}
\label{sec:org6af66e5}

In R there are several pre-loaded data

\begin{verbatim}
## In case the code below does not work
## you may need to install the package dataset
## For a list of available  datasets, type library(help = "datasets")
head(iris)
\end{verbatim}

\begin{center}
\begin{tabular}{rrrrl}
Sepal.Length & Sepal.Width & Petal.Length & Petal.Width & Species\\
\hline
5.1 & 3.5 & 1.4 & 0.2 & setosa\\
4.9 & 3 & 1.4 & 0.2 & setosa\\
4.7 & 3.2 & 1.3 & 0.2 & setosa\\
4.6 & 3.1 & 1.5 & 0.2 & setosa\\
5 & 3.6 & 1.4 & 0.2 & setosa\\
5.4 & 3.9 & 1.7 & 0.4 & setosa\\
\end{tabular}
\end{center}


We can also obtain a summary of the data set
\begin{verbatim}
summary(iris)
\end{verbatim}

\begin{center}
\begin{tabular}{lllll}
Sepal.Length & Sepal.Width & Petal.Length & Petal.Width & Species\\
\hline
Min.   :4.300 & Min.   :2.000 & Min.   :1.000 & Min.   :0.100 & setosa    :50\\
1st Qu.:5.100 & 1st Qu.:2.800 & 1st Qu.:1.600 & 1st Qu.:0.300 & versicolor:50\\
Median :5.800 & Median :3.000 & Median :4.350 & Median :1.300 & virginica :50\\
Mean   :5.843 & Mean   :3.057 & Mean   :3.758 & Mean   :1.199 & nil\\
3rd Qu.:6.400 & 3rd Qu.:3.300 & 3rd Qu.:5.100 & 3rd Qu.:1.800 & nil\\
Max.   :7.900 & Max.   :4.400 & Max.   :6.900 & Max.   :2.500 & nil\\
\end{tabular}
\end{center}



\section{Writing a data frame}
\label{sec:orgba122d6}

Write a data frame by specifying the columns:

\begin{verbatim}
df <- data.frame(
"NAME" =  c("Aleksandra", "Hugo", "Piotr", "Ewa"),
"AGE"  =  c(29,35, 39, 33),
"HEIGHT"= c(1.68, 1.83, 2.03, 1.66) )
df
\end{verbatim}

\begin{center}
\begin{tabular}{lrr}
NAME & AGE & HEIGHT\\
\hline
Aleksandra & 29 & 1.68\\
Hugo & 35 & 1.83\\
Piotr & 39 & 2.03\\
Ewa & 33 & 1.66\\
\end{tabular}
\end{center}


What is the dimension of the data frame?
\begin{verbatim}
dim(df)
\end{verbatim}

What are the variable names of the data frame?
\begin{verbatim}
names(df)
\end{verbatim}



\section{Reading a data frame}
\label{sec:org403cb10}

From a .csv file

\begin{verbatim}
df <- read.csv('./path_to/file.csv')
\end{verbatim}

From a .dbf file
\begin{verbatim}
library(foreign)
df <- read.csv('./path/to/file.dbf')
\end{verbatim}

From a .sav file
\begin{verbatim}
library(foreign)
     df <- read.spss(file='./path/to/file.sav', to.data.frame=TRUE) 
     str(df)   # show the structure of the data frame
\end{verbatim}

From the clipboard
\begin{verbatim}
df2 <- read.table(file = "clipboard", sep = "\t", header=TRUE)
\end{verbatim}


\section{Creating a new variable}
\label{sec:org777e5fa}

\begin{itemize}
\item Graphics
\item Reading data
\item Markdown
\end{itemize}
\end{document}